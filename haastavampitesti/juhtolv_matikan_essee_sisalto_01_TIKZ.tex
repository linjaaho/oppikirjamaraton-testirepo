


\part{Puhtauspalvelualan matematiikkaa}

\chapter{Foo}

% Ja laitetaan toki vielä lähde, kun CC-lisensseillä mennään:
% http://www.texample.net/tikz/examples/cuboid/
% Author: Florian Lesaint (CC-BY)
\begin{tikzpicture}
	%%% Edit the following coordinate to change the shape of your
	%%% cuboid
      
	%% Vanishing points for perspective handling
	\coordinate (P1) at (-7cm,1.5cm); % left vanishing point (To pick)
	\coordinate (P2) at (8cm,1.5cm); % right vanishing point (To pick)

	%% (A1) and (A2) defines the 2 central points of the cuboid
	\coordinate (A1) at (0em,0cm); % central top point (To pick)
	\coordinate (A2) at (0em,-2cm); % central bottom point (To pick)

	%% (A3) to (A8) are computed given a unique parameter (or 2) .8
	% You can vary .8 from 0 to 1 to change perspective on left side
	\coordinate (A3) at ($(P1)!.8!(A2)$); % To pick for perspective 
	\coordinate (A4) at ($(P1)!.8!(A1)$);

	% You can vary .8 from 0 to 1 to change perspective on right side
	\coordinate (A7) at ($(P2)!.7!(A2)$);
	\coordinate (A8) at ($(P2)!.7!(A1)$);

	%% Automatically compute the last 2 points with intersections
	\coordinate (A5) at
	  (intersection cs: first line={(A8) -- (P1)},
			    second line={(A4) -- (P2)});
	\coordinate (A6) at
	  (intersection cs: first line={(A7) -- (P1)}, 
			    second line={(A3) -- (P2)});

	%%% Depending of what you want to display, you can comment/edit
	%%% the following lines

	%% Possibly draw back faces

	\fill[gray!90] (A2) -- (A3) -- (A6) -- (A7) -- cycle; % face 6
	\node at (barycentric cs:A2=1,A3=1,A6=1,A7=1) {\tiny f6};
	
	\fill[gray!50] (A3) -- (A4) -- (A5) -- (A6) -- cycle; % face 3
	\node at (barycentric cs:A3=1,A4=1,A5=1,A6=1) {\tiny f3};
	
	\fill[gray!30] (A5) -- (A6) -- (A7) -- (A8) -- cycle; % face 4
	\node at (barycentric cs:A5=1,A6=1,A7=1,A8=1) {\tiny f4};
	
	\draw[thick,dashed] (A5) -- (A6);
	\draw[thick,dashed] (A3) -- (A6);
	\draw[thick,dashed] (A7) -- (A6);

	%% Possibly draw front faces

	% \fill[orange] (A1) -- (A8) -- (A7) -- (A2) -- cycle; % face 1
	% \node at (barycentric cs:A1=1,A8=1,A7=1,A2=1) {\tiny f1};
	\fill[gray!50,opacity=0.2] (A1) -- (A2) -- (A3) -- (A4) -- cycle; % f2
	\node at (barycentric cs:A1=1,A2=1,A3=1,A4=1) {\tiny f2};
	\fill[gray!90,opacity=0.2] (A1) -- (A4) -- (A5) -- (A8) -- cycle; % f5
	\node at (barycentric cs:A1=1,A4=1,A5=1,A8=1) {\tiny f5};

	%% Possibly draw front lines
	\draw[thick] (A1) -- (A2);
	\draw[thick] (A3) -- (A4);
	\draw[thick] (A7) -- (A8);
	\draw[thick] (A1) -- (A4);
	\draw[thick] (A1) -- (A8);
	\draw[thick] (A2) -- (A3);
	\draw[thick] (A2) -- (A7);
	\draw[thick] (A4) -- (A5);
	\draw[thick] (A8) -- (A5);
	
	% Possibly draw points
	% (it can help you understand the cuboid structure)
	\foreach \i in {1,2,...,8}
	{
	  \draw[fill=black] (A\i) circle (0.15em)
	    node[above right] {\tiny \i};
	}
	% \draw[fill=black] (P1) circle (0.1em) node[below] {\tiny p1};
	% \draw[fill=black] (P2) circle (0.1em) node[below] {\tiny p2};
\end{tikzpicture}

\begin{align*}
2,05\ l = x\ dm^3  &\Leftrightarrow 1\ l = 1\ dm^3 \\
&\implies x = 2,05 \\
&\implies  2,05\ l = 2,05\ dm^3
\end{align*}


\begin{align*}
1\ l &= 1000\ ml \\
\frac{5400}{1000} &= 5,4 \\
\implies 5400\ ml &= 5,4\ l
\end{align*}



\begin{align*}
1\ cm^2 &= 100\ mm^2 \\
100 \cdot 300 &= 30000 \\
\implies 300\ cm^2 &= 30000\ mm^2
\end{align*}


\begin{align*}
100\ cm &= 1\ m \\
\frac{7}{100} &= 0,07 \\
\implies 7\ cm &= 0,07\ m \\
\implies 2\ m + 7\ cm &= 2\ m + 0,07\ m = 2,07\ m
\end{align*}


\begin{align*}
1\ min &= 60\ s \\
\frac{1500}{60} &= 25 \\
\implies 1500\ s &= 25\ min
\end{align*}


\begin{align*}
1\ h &= 60\ min \\
1,35 \cdot 60 &= 81 \\
\implies 1,35\ h &= 81\ min
\end{align*}


\begin{align*}
1\ l &= 10\ dl = 100\ cl \\
\frac{8}{10} &= 0,8 \\
\frac{8}{100} &= 0,08 \\
\implies 8\ dl + 8\ cl &= 0,8\ l + 0,08\ l = 0,88\ l
\end{align*}


\begin{align*}
1\ ha &= 10000\ m^2 \\
5 \cdot 10000 &= 50000 \\
\implies 5\ ha &= 50000\ m^2
\end{align*}


\chapter{Bar}

\begin{align*}
\frac{5}{475} &= 0,010526 \\
              &=  01,0526\ \%
\end{align*}



\[
180 + \left ( \frac{180}{100} \cdot 15 \right ) = 207
\]




\[
\frac{15\ kg}{20} \cdot 100 = 75\ kg
\]




\[
\frac{75}{250} = 0,3 = 30\ \%
\]


\chapter{Baz}



\begin{align*}
\frac{1500\ \text{€}}{100 + 3,5} \cdot 3,5 &= 50,724635\ \text{€}  \\
                                 &\approx 50,72\ \text{€}
\end{align*}




\begin{align*}
1500\ \text{€} - \left ( \frac{1500\ \text{€}}{100 + 3,5}  \cdot 3,5 \right ) &= 1500\ \text{€} - 50,724635\ \text{€} \\
&= 1449,275365\ \text{€}\\
&\approx 1449,28\ \text{€}
\end{align*}






\begin{align*}
1\ l &= 1000\ ml \\
\frac{5\ \text{€}}{1000} \cdot 750 &= 3,75\ \text{€}
\end{align*}




\begin{align*}
1\ l &= 10\ dl \\
\frac{41,50\ \text{€}}{5\ l} &= 8,30\ \text{€}/l \\
\frac{2,3}{10} = 0,23 \implies 2,3\ dl &= 0,23\ l \\
7\ d \cdot 0,23\ l \cdot 8,30\ \text{€}/l &= 7\ d \cdot 1,9090\ \text{€} \\
&= 13,3630\ \text{€}/viikko \\
&\approx 13,36\ \text{€}/viikko
\end{align*}


\chapter{Qux}


\begin{align*}
25\ min + 3600\ s + 3,5\ h &= \\
25 \cdot 60\ s + 3600\ s + 3,5 \cdot 60 \cdot 60\ s &= \\
1500\ s + 3600\ s + 12600\ s &= 17700\ s \\
1\ h &= 3600\ s \\
\frac{17700}{3600} &= 4,916666 \\
\implies 17700\ s &= 4,916666\ h
\end{align*}




\begin{align*}
45\ min + 7,2\ h + 5,6\ h - 2,1\ h &= \\
\frac{45}{60}\ h + 7,2\ h + 5,6\ h - 2,1\ h &= 11,45\ h
\end{align*}




\begin{align*}
23\ h + 7,8\ h + 0,7\ h = 31,5\ h
\end{align*}




\begin{align*}
3,3\ h + 45\ min - 2,6\ h &= \\
3,3\ h + \frac{45}{60}\ h - 2,6\ h &= 1,45\ h
\end{align*}


\chapter{Quux}



\begin{align*}
5\ d \cdot 40 \cdot 20\ ml &= 4000\ ml/viikko \\
                           &= 4\ l/viikko
\end{align*}




\begin{align*}
5\ d \cdot 40 \cdot 5\ ml &= 1000\ ml/viikko \\
                          &= 1\ l/viikko \\
\left ( \frac{365\ d}{7\ d} \right ) \cdot 1\ l &= 52.142857\ l/a \\
\end{align*}




\begin{align*}
52.142857\ l \cdot 3,20\ \text{€}/l &= 166,857142\ \text{€} \\
                                    &\approx 166,86\ \text{€}
\end{align*}


\chapter{Quuux}



\begin{align*}
250\ m^2 \cdot 0,18\ min/m^2 \cdot 260\ d &= 11700\ min \\
                                          &= 195\ h
\end{align*}




\begin{align*}
250\ m^2 \cdot \frac{260\ d}{2} \cdot 0,009\ min/m^2 &= 292,5\ min \\
                                                     &= 4,875\ h
\end{align*}


\part{Erilaisia sokkotekstejä}

%%% Erilaisia sokkotekstejä (Lorem ipsumia)

%%% blindtext-paketista

%\blinddocument
\Blinddocument

\chapter{Sokkotekstiä}

\section{Sokkotekstiä oletusfontilla}

\Blindtext

\section{Sokkotekstiä kirjoituskoneella}

\texttt{\blindtext}


\section{Sokkotekstiä sans-seriffillä}

\textsf{\blindtext}

\section{Sokkotekstiä seriffillä}

\textrm{\blindtext}



%%% Ei toimi tässä
%\blindmathpaper

\chapter{Lipsum}

%%% lipsum-paketista

\section{Lorem ipsum}

\lipsum[1-5]

\texttt{\lipsum[6-10]}

\textsf{\lipsum[11-15]}

\textrm{\lipsum[16-20]}

%%% Local Variables: 
%%% mode: latex
%%% End: 
